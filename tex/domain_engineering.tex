%Figure~\ref{fig:proline} depicts a screenshot of ProLine-RM. 
%The DE phase is exhibited by letter ``A'' to ``C'' in Figure~\ref{fig:proline}. 
%In order to illustrated the useful of the ProLine-RM and its functionalities we describe all activities necessaries to devise a ``persistence'' CFF. 
As in development of any framework, firstly the expertise of the domain has to be got.
Therefore, the domain related to ``persistence'' has been studied. 
The outcome of such study were the identification of both the common features and its variants of the domain. 

%After identifying the all features of the CFF the next activity is the development of the CFF. 
In this activity all source code are developed and organized in packages. Each package contain source code (e.g., classes, aspects and methods) related to a feature. %For the purpose of accomplishing this activity we have used the approach described by~\citet{deCamargo:2008:PDC:1363686.1363863}, its aims is to assists and makes easier the develop of the CFF by using aspect oriented paradigm~\cite{Kiczales97aspect-orientedprogramming}. 

Afterwards, the feature model depicting all features related to the domain has to be modeled. 
Aiming to make easier the development of feature models, ProLine-RM provides a graphical way to assists the engineer devises them. 
Figure~\ref{fig:proline}(C) shows the feature model that we have developed. 
As can be seen, there are two set of mandatory features and two groups related to optional features as well.  
%The first one, called ``Persistence'' aims to introduce a set of persistence operations into application persistence classes (e.g., store, remove, update, perform queries). 
%The second feature, named ``Connection'' is related to the database connection concern and identifies points in the application code where the connection must be opened and closed. 
%This feature has variabilities, as for example Data Base Management System (e.g., MySQL, SyBase, Native and Interbase).
%There are two optional features as well. The former is called ``Caching'', which is responsible to deals with high-performance to gets datas of the databases.
%The second, named ``Pooling'' is represented a set of database connections maintained by the databases.

After developing the CFF and the its feature model, them have to be uploaded in a remote repository in order to be reused during the AE phase.
Prior to uploading these artifacts, informations (e.g., Named of CFF, Author(s) and Description) associated with the ``persistence'' CFF has to be filled in. Figure~\ref{fig:proline}(A) shows an example, which the CFF and its feature model are being uploaded. In the next section is described how to reuse the ``persitence'' CFF.