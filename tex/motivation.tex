The motivation is the following: (\textit{i}) there is neither a repository, in which the CFs/CFFs can be stored by domain engineer in order to share it, nor an infrastructure where the CFs can be reused as easy as possible in order to disseminate them; (\textit{ii}) most of the CFs found apply white-box reuse strategies - thus it is important provides a way to assist the instantiation of these CFs using models.     

%(\textit{i}) provide an infrastructure to support the reuse of CFs, (\textit{ii}) allowing application engineers to download CF members to be reused in their applications and (\textit{iii}) allowing application engineers to conduct the reuse of CFs in a model-based fashion.  


%In the literature, it is possible find out a large number of researches related to CFF (COLOCAR REF). In these researches a large number of CFF have been devised (e.g., Security, Persistence, Distribution, etc). Thus, it is important develop an environment in which allows engineers share, manage, and provide full cycle of reuse of these CFF. It is also important supplies a graphical way to allow the engineer examines and reuses the features available in one CFF. For instance, the engineer could reuse piece of code of the CFF just selecting a set of features through a feature model.

%Therefore, this environment would increases the quality of the base applications that are developed in terms of modularity, reuse, maintainability. The efficiency will be achieved through the use of techniques related to reuse of analysis, design and code. The quality of the final applications must also be improved with the reuse, since the artifacts have already been tested and approved.  


%For instance, one engineer has developed a CFF and would like to share it in oder to let another engineer reuse his CFF.  CFF has beed developed and  engineer developed a CFF Moreover, it is also important supplies a graphical way to allow the engineer examines and reuses the features available in one CFF. For instance, the engineer could reuse piece of code of the CFF by selecting a set of features in a feature model. % if them fulfill the base application.  the features available fulfilled the base application the engineer would reuse such features.  %whether the features available in one CFF fulfill the base application requirements. 
%To the best of our knowledge there are neither an approach nor a tool with these functionality. In order to overcome such limitations we have put forward a plug-in, in which is described in the next sections.	 

%However, to the best of our knowledge, there is neither an approach nor a tool that allows share, management and also provides full cycle of reuse of the CFF. Moreover, it is important a tool that supplies a graphical way to allow the engineer examines previously if the features available in one CFF fulfill the application requirements. %Moreover, it is important that there is way to allow the engineer verify whether the variability/features available in the CFF meet the application requirements that must be developed.  

%Afterwards, the features have been examined and chosen by the engineer it is likewise important provides a way to download only the artifacts that have been chosen.    

%Thus, the CFF can be used during a development process easier and can be provided major computer support to facilitate such as task.     

%It is important that there is a way to examine whether the variability / features available in the family of FTs meet the application requirements that must be developed. In order for families of FTs to be properly used during a development process, and a major computer support to facilitate this task.


% which implements complete cycle of reuse of the CFF. 
%\textbf{Using this plug-in the CFF can be reused in a controlled way. Thus, during the reuse to increase the quality of the developed applications. Furthermore, any manipulation in the CFF can be made through the intermediary of models, so we intend to raise the level of abstraction in which the engineer works.}


%Thus, the objective of this paper is describes this plug-in. 

%Embora v�rios autores j� tenham trabalhado com FTs (Couto et al., 2005; Hanenberg et al., 2004; Huang et al., 2004; Shah e Hill, 2004; Constantinides e Elrad, 2001; Pinto et al., 2002; Rashid e Chitchyan, 2003; Soares et al., 2006; Vanhaute et al., 2001; Huang et al., 2004; Mortensen e Ghosh, 2006a; Mortensen e Ghosh, 2006b; Soares, 2004), n�o � encontrado na literatura o projeto de um reposit�rio de FTs e t�cnicas para busca de um determinado FT que atenda aos requisitos de uma aplica��o. � importante que exista uma forma de averiguar se as variabilidades/caracter�sticas dispon�veis na fam�lia de FTs atendem aos requisitos da aplica��o que deve ser desenvolvida. Para que fam�lias de FTs sejam adequadamente usadas durante um processo de desenvolvimento, � importante um apoio computacional que facilite essa tarefa.