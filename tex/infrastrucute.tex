Figure~\ref{fig:process} depicts the overall process to use KDM-RE - which was adapted based on the horseshoe modernization model. 
It is split into three steps, they are: 
(\textit{i}) reverse engineering, 
(\textit{ii}) refactorings, and 
(\textit{iii}) forward engineering. 
Furthermore, these steps were divided into six levels, as can be seen in Figure~\ref{fig:process} - more details about these steps and its levels are described in next sections:

\begin{figure}[!ht]
\centering
  % Requires \usepackage{graphicx}
  \includegraphics[width=11cm, height=3.8cm]{figure/processoDaFerramenta}
\caption{KDM-RE Process}
\label{fig:process}
\end{figure} 

\subsection{Reverse Engineering}

Herein the engineer provides a KDM file or a source-code from a legacy system. If the source code is the input, then the engineer starts the process in the \textbf{Level-0} by choosing an eclipse project which contains the source-code to realize the refactorings.  
After that, into \textbf{Level-1} the source-code need to be transformed into a Platform-Specific Model (PSM). 
This PSM is an instance of the source-code which represents all abstraction of the source-code. To realize this transformation we implemented a model extractor in Java. 
Figure~\ref{fig:discovery_java_model} illustrates how KDM-RE manages to assist the engineer to get the instance of this PSM. Figure~\ref{fig:discovery_java_model} (a) shows the eclipse project selected by the engineer - then after the engineer click in a popup menu named ``Discovery Java Model'' the Figure~\ref{fig:discovery_java_model} (b) is shown an excerpt to indicate the correspondence with the legacy Java Model. For instance, each ``Class'' found in the source-code an instance of the meta-classe \texttt{ClassDeclaration} is created, similar each ``Methods'' declarations are transformed to instances of the meta-classe \texttt{MethodDeclaration}, and ``attributes'' are transformed into instances of \texttt{FieldDeclaration}, etc.
%This PSM is an Abstract Syntax Tree (AST) that represents the syntactic struct of Java source code.



After creating the PSM the next level (\textbf{Level-2}) consists in transforming the PSM to a Platform-Indented Model (PIM) which is based on the KDM. 
In this level the KDM-RE uses MoDisco\url{http://www.eclipse.org/MoDisco/} which provides an extensible framework to transform a specific PSM to KDM models in order to represent the systems ``AS-IS''. We added in KDM-RE a popup menu named ``Discovery KDM Model'' which by clicking on it the KDM-RE calls the MoDisco API to get a instance of the KDM based on the earlier PSM.
In Figure~\ref{fig:discovery_java_model}(c) shows how the Java Model is transformed to KDM.
%
\begin{figure}[!ht]
\centering
  % Requires \usepackage{graphicx}
  \includegraphics[width=13cm, height=6cm]{figure/GerandoTODOS}
\caption{Process to Discovery Java and KDM Model}
\label{fig:discovery_java_model}
\end{figure}
%
In Figure~\ref{fig:interface} we depicted the main window of our \textit{plug-in}. 
For explanation purpose, we have identified main regions, i.e., \textcircled{a}, \textcircled{b}, \textcircled{c} and \textcircled{d}.

\begin{figure}[!ht]
\centering
  % Requires \usepackage{graphicx}
  \includegraphics[width=15cm, height=6.8cm]{figure/ScreenShot_tool}
\caption{KDM-RE's Interface}
\label{fig:interface}
\end{figure}

All refactorings provided by KDM-RE are based on the KDM model. 
%Thus, suppose that a KDM model has already been instantiated. 
%All the steps to how obtain a KDM instance are explained further. 
In order to assist the refactorings we extended the KDM's model browser provided by MoDisco. 
We added a popup menu named ``Refactoring KDM'' in this model browser, see Figure~\ref{fig:interface}\textcircled{a}.
By using this menu the engineer can interact with the KDM model and choose which refactoring must be carried out in the KDM.
In the region \textcircled{a} can be seen all 16 refactorings that have been implemented in KDM-RE. 
For illustration purposes only we drew rectangles to separate the refactorings into three groups. 
The black rectangle represents refactorings that deal with generalization, the blue rectangle stand for refactorings to organize data and the red one symbolize refactoring to assist the moving features between objects.

The region \textcircled{b} on Figure~\ref{fig:interface} shows a class diagram that can be used either before to apply some refactorings in order to assist the engineer to decide where to apply the refactorings or this class diagram can be generated as the engineer performs the refactorings in KDM model, i.e., changes are reproduced on the fly in a class diagram.
We claim that the latter use of the class diagram is important once the class diagram provides an abstract view of the system, hence, the engineer can visually check the system's changes after applying a set of refactorings. 
In addition, usually the source code is the only available artifact of the legacy software. 
Therefore, creating a class diagram makes, both the legacy software and the generated software to have a new type of artifact (i.e., UML class models), improving their documentation.

KDM-RE also supplies a multiple versions of a system at level models (KDM) which allows the engineer to work interactively on multiple models and to explore alternate refactoring path. As can see in the region \textcircled{c} (see Figure~\ref{fig:interface}), the engineer must select a KDM file, then he must right-click on the mouse to appear a popup menu named ``Versions''. By releasing the mouse on this menu, three options is shown: (\textit{i}) List of changes, (\textit{ii}) Delete version and (\textit{iii}) Create a child version. The last option create a copy of the KDM file - then the engineer can explore another refactoring path. The second option delete a specific version - first option shows all changes that have been realized in a current KDM file, all changes are depicted in an Eclipse View, as shown in region \textcircled{d}. In this View it is possible to visualize the author that have committed the changes, the project and all refactorings realized.


\subsection{Executing Refactoring KDM-RE}

After the engineer clicks on the menu-item in region \textcircled{a} upon Figure~\ref{fig:interface} and choose which refactoring to apply, a method \texttt{run()} in the class related to the chosen refactoring is being called. In this method the refactoring classes and a ``RefactoringWizard'' are started. In every refactoring the KDM file must be analyzed to find the meta-classes that are affected by a specific refactoring. Because of the structure of the KDM file, the easiest way to do this, is a traversal using the visitor pattern~\cite{Gamma1994}. Therefore we implemented the visitor pattern in KDM-RE to assist the travel of all meta-classes in a correct order.

\begin{figure}[!ht]
\centering
  % Requires \usepackage{graphicx}
  \includegraphics[scale=0.6]{figure/Wizard2}
\caption{Extract Class Wizard}
\label{fig:wizard}
\end{figure}

Notice that KDM-RE also provides a way to indicate refactoring opportunities. Refactoring itself will not bring the full benefits, if we do not understand when refactoring needs to be applied. Thus, to make it easier for the engineer to decide whether certain software needs refactoring or 
not KDM-RE implements a set of bad code smell according to~\cite{bad_smeel_}.

For explanation purpose pretend that the KDM-RE found out that one class is doing work that should be done by two, thus, he must apply the refactoring ``Extract Class''. The first step is to select the metaclass that KDM-RE identified as a bad smell, i.e., the metaclass to be extracted into a separate one, this step is illustrated in Figure~\ref{fig:wizard}(a). After selecting the metaclass, a right-click opens the context menu where the refactoring is accessible. After the click, the system displays the ``RefactoringWizard'' to the engineer, Figure~\ref{fig:wizard}(b) depicts the Extract Class Wizard. In this wizard, the name of the new metaclass can be set. Also a preview of all detected \texttt{StorableUnits} and \texttt{MethodUnits} that can be chosen to put into the new metaclass. Further, the engineer can select if either the new metaclass will be a top level metaclass or a nested metaclass. The engineer also can select if the KDM-RE must create instances of \texttt{MethodUnits} to represent accessors methods (gets and sets). Finally, the engineer can set the name of the \texttt{StorableUnit} that represent the link between the two metaclasses (the old metaclass and the new one). After all of the required inputs have been made, the engineer can click on the button ``Finish'' and the refactoring ``Extract Class'' is performed by KDM-RE. As can be seen in Figure~\ref{fig:wizard}(c) a new instance of \texttt{ClassUnit} named ``Document'' was created - two \texttt{StorableUnit} from ``Pessoa'', i.e., ``rg'' and ``CPF'' were moved to the new \texttt{ClassUnit} - instances of \texttt{MethodUnits} were also created to represent the gets and sets. In addition, the instance of \texttt{ClassUnit} named ``Pessoa'' owns a new instance of \texttt{StorableUnit} that represent the link between both \texttt{ClassUnits}. Due space limitation the other \texttt{StorableUnits} of \texttt{ClassUnit} named ``Pessoa'' are not shown in Figure~\ref{fig:wizard}(c).

After the engineer realize the refactoring an UML class diagram is created on the fly to mirror graphically all changes performed in the KDM model, see Figure~\ref{fig:interface}\textcircled{b}. Moreover, the KDM-RE creates/updates a tracking log to show the historic of all changes performed in the system, as can be see in Figure~\ref{fig:interface}\textcircled{d}. 

\subsection{Forward Engineering}
After the engineer realize the refactoring the next step are to transform the KDM model to a PSM, i.e., a Java Meta-Model and to generate the refactored source-code conforming the PSM. The former step is carried out based on a set of transformations using ATL, due space limitation these transformations are not depicted. Then after transform the KDM to a instance of Java meta-model 

The latter was performed by using a textual template approach, such as the Acceleo\footnote{http://www.eclipse.org/acceleo/}. A template can be thought of as the target text with holes for variable parts. The holes contain metacode which is run at template instantiation time to compute the variable parts. In Figure~\ref{fig:forward_Engineering} the generation of source code using template is depicted.

\begin{figure}[!ht]
\centering
  % Requires \usepackage{graphicx}
  \includegraphics[scale=0.48]{figure/Forward_Engineering2}
\caption{Forward Engineering steps}
\label{fig:forward_Engineering}
\end{figure}