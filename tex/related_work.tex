Borger et al.~\cite{Boger2002} developed a plug-in for the CASE tool ArgoUML that support UML model-based refactorings. The refactoring of class, states and activities is possible, allowing the user to apply refactorings that are not simple to apply at source code level.
Van Gorp et al.~\cite{Gorp03towardsautomating} proposed a UML profile to express pre and post conditions of source code refactorings using Object Constraint Language (OCL) constraints. The proposed profile allows that a CASE tool: (\textit{i}) verify pre and post conditions for the composition of sequences of refactorings; and (\textit{ii}) use the OCL consulting mechanism to detect bad smells such as crosscutting concerns.  Reimann et al.~\cite{Models2010} present an approach for EMF model refactoring. They propose the definition of EMF-based refactoring in a generic way. Another approach for EMF model refactoring is presented in~\cite{ModelsEMFREfactorin}. They propose EMF Refactor~\footnote{http://www.eclipse.org/emf-refactor/}, which is a new Eclipse incubation project in the Eclipse Modeling Project consisting of three main components. Besides a code generation module and a refactoring application module, it comes along with a suite of predefined EMF model refactorings for UML and Ecore models.

%The differential of our approach described herein in relation to the other is the proposal to move from software refactoring to model-driven refactoring by means of KDM, which is a platform and language independent meta-model. 