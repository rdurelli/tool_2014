In this paper is presented the KnowDIME to support the refactoring of legacy systems based on ADM, which uses the KDM standard. It follows the theory of the horseshoe modernization model, which is threefold: (\textit{i}) \textbf{Reverser Engineering}, (\textit{ii}) \textbf{Reestructuring} and  (\textit{iii}) \textbf{Forward Engineering}. 

Firstly, all the artefacts of the legacy system must be transformed into PSMs by statically analyzing the legacy source code. Still in the first step, these PSMs are integrated into a KDM model, i.e., a PIM model, through a M2M transformation implemented using ATL. Secondly, the KnowDIME applies a set of model refactorings and model optimizations in this PIM. Afterwards, KnowDIME executes a set of M2M transformation taking as input the PIM and producing as output a model conforming to the KDM models into UML meta model. Finally, an improved system is obtained from this UML by means of a set of M2C transformation; additionally, the generated code can be complemented by the software engineer in accordance with the more detailed specifications of the application business rules, such as the implementation of specific behaviors and features not covered by the code generation.

We believe that KnowDIME makes a contribution to the challenges of Software Engineering which focuses on mechanisms to support the automation of software refactoring process. Long term future work involves conducting experiments to evaluate the level of maintenance provided by KnowDIME. It is worth highlighting that KnowDIME is open source and it can be downloaded at\textit{~www.dc.ufscar.br/$\sim$valter/crossfire}.
