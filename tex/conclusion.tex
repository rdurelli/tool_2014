In this paper is presented the KDM-RE which provides support to model-driven refactoring based on ADM and uses the KDM standard. It follows the theory of the horseshoe modernization model, which is threefold: (\textit{i}) Reverser Engineering, (\textit{ii}) Refactoring and  (\textit{iii}) Forward Engineering. 

Firstly, the engineer starts the refactoring process by means of a KDM file or a source code to be refactored. If the engineer provides a KDM file then KDM-RE can already apply the refactorings. Otherwise, the source code of the legacy system must be transformed into PSMs. Still in the first step, these PSMs are converted into a KDM model through a set of M2M transformation by means of MoDisco. Secondly, the engineer by using KDM-RE can apply a set refactorings in this KDM. Also, on the fly the engineer can check all changes realized in this KDM replicated into a class diagram - the engineer can visually verify the system`s changes after applying a set of refactorings. KDM-RE also supplies a multiple versions of a system at level models. Finally, KDM-RE performs a forward engineering then a refactored source code is generated.

We believe that KDM-RE makes a contribution to the challenges of Software Engineering which focuses on mechanisms to support the automation of model-driven refactoring. Future work involves implementing more refactorings and conducting experiments to evaluate all refactorings provided by KDM-RE. Notice that KDM-RE is open source and it can be downloaded at\textit{~www.dc.ufscar.br/$\sim$valter/KDM-RE}.
