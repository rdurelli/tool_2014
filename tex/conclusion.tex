In this paper is presented the KDM-RE which is a \textit{plug-in} on the top of the Eclipse Platform to provide support to model-driven refactoring based on ADM and uses the KDM standard. More specifically, this \textit{plug-in} supports 17 refactorings adapted to KDM. These refactorings are based on some fine-grained refactorings proposed by Fowler~\cite{refactImpro}. As stated in the case study the engineer/modernizer by using KDM-RE can apply a set refactorings in a KDM. Also, on the fly the engineer can check all changes realized in this KDM replicated into a class diagram - the engineer can visually verify the system`s changes after applying a set of refactorings. In addition, usually the source code is the only available artifact of the legacy software. 
Therefore, creating an UML class diagram makes, both the legacy software and the generated software to have a new type of artifact (i.e., UML class models), improving their documentation. %KDM-RE also supplies a multiple versions of a system at level models. 
Also, we claim that as we have defined all refactoring based on the KDM, they can be easily reused by others researchers. 

% It follows the theory of the horseshoe modernization model, which is threefold: (\textit{i}) Reverser Engineering, (\textit{ii}) Refactoring and  (\textit{iii}) Forward Engineering. 

%Firstly, the engineer starts the refactoring process by means of a KDM file or a source code to be refactored. If the engineer provides a KDM file then KDM-RE can already apply the refactorings. Otherwise, the source code of the legacy system must be transformed into PSMs. Still in the first step, these PSMs are converted into a KDM model through a set of M2M transformation by means of MoDisco. Secondly, the engineer by using KDM-RE can apply a set refactorings in this KDM. Also, on the fly the engineer can check all changes realized in this KDM replicated into a class diagram - the engineer can visually verify the system`s changes after applying a set of refactorings. KDM-RE also supplies a multiple versions of a system at level models. Finally, KDM-RE performs a forward engineering then a refactored source code is generated.

It is important to notice that the application of refactorings in UML class diagrams is not a new research as stated before. However, all of the works we found on literature perform the refactoring directly on the UML metamodel. Although UML is also an ISO standard, its primary intention is just to represent diagrams and not all the characteristics of a system.  As KDM has been created to represent all artifacts and all characteristics of a system, refactorings performed on its finer-grained elements can be propagated to higher level elements. This propitiates a more complete and manageable model-driven modernization process because all information is concentrated in just one metamodel. 
In terms of the the users who uses modernization tools like ours, the difference is not noticeable; that is, whether the refactorings are performed over UML or KDM. However, there are two main benefits of developing a refactoring catalogue for KDM. The first one is in terms of reusability. Other  modernizer engineers can take advantage of our catalogue to conduct modernizations in their  systems. The second benefit is that, unlikely UML, a catalogue for KDM can be extended to higher abstractions levels, such as architecture and conceptual, propitiating a good traceability among  these layers. 

We believe that KDM-RE makes a contribution to the challenges of Software Engineering which focuses on mechanisms to support the automation of model-driven refactoring. Future work involves implementing more refactorings and conducting experiments to evaluate all refactorings provided by KDM-RE. Doing so, we hope to address a broader
audience with respect to using, maintaining, and evaluating
our tools. Currently, KDM-RE generates only class diagrams to assist the modernization engineer to perform refactorings, however, as future work, we intend to: (\textit{i}) extend this computational support to enable the achievement of other diagrams, e.g.,  the sequence diagram, (\textit{ii}) perform structural check of the software after the application of refactorings; and (\textit{iii}) carry out the assessment tool, as well as refactorings proposed by controlled experiments. %Notice that KDM-RE is open source and it can be downloaded at\textit{~www.dc.ufscar.br/$\sim$valter/KDM-RE}.
%
A work that is already underway is to check how other parts of the highest level of KDM are affected after the application of certain refactorings. For example, assume that there are two packages P1 and P2. Suppose there is a class in P1, named C1, and within the P2 there is a class named C2. Assume that C1 owns an attribute A1 of the type C2., i.e., there is an association relationship between these classes of different packages. P1 and P2 represent architectural layers, i.e., P1 = Model and P2 = View. Thus, the relationship that exists is undesirable. When we make a fine-grained refactoring such as moving the attribute A1 of the class C1, it should be reflected to the architectural level, eliminating the existing relationship between the two architectural layers.
