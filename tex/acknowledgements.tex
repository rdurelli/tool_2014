The authors would like to thank FAPESP for Process 2012/05168-4.