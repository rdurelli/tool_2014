
\documentclass[12pt]{article}

\usepackage{sbc-template}
\usepackage{graphicx,url}
\usepackage{array}
\usepackage{color}
\usepackage{listings}
\usepackage{fancyhdr}
\usepackage[T1]{fontenc}
\usepackage{epsfig}
\usepackage{rotating}
\usepackage{setspace}

\usepackage{graphicx}

\usepackage[square,
            authoryear,
            sort&compress,]{natbib}
\let\cite=\citep

%\usepackage[brazilian]{babel}
\usepackage[latin9]{inputenc}

     
\sloppy
\newcommand{\papertitle}{KnowDIME: An Infrastructure based on Architecture-Driven Modernization for Improving Legacy System}
\title{\papertitle}

\author{Rafael S. Durelli\inst{1}, Bruno  M. Santos\inst{2}, Raphael R. Honda\inst{2}, \\  M\'{a}rcio E. Delamaro\inst{1} and Valter V. de Camargo\inst{2}}
\address{Computer  Systems Department University of S\~{a}o Paulo - ICMC\\
  S\~{a}o Carlos, SP, Brazil.
\nextinstitute
  Computing Departament\\ Federal University of S\~{a}o Carlos - UFSCAR\\
  S\~{a}o Carlos, SP, Brazil.
%\nextinstitute
 % RMoD Team, INRIA, Lille, France
  \email{\{rdurelli, delamaro\}@icmc.usp.br\inst{1}, valter@dc.ufscar.br\inst{2}}
}

\begin{document} 

\maketitle

\begin{abstract}


%ABSTRACT 1
Software refactoring is a proven technique that aims at improving the quality of source code.
 However, nowadays with with the advent of Architecture-Driven Modernization (ADM), 
which follows the Model-Driven Architecture (MDA) principles and uses the Knowledge Discovery Metamodel (KDM), 
it seems a good alternative for moving from software refactoring to model-driven refactoring in order to create a language and platform independent catalog of refactoring. 
Nevertheless, although ADM provides the process for refactoring legacy systems, is so far missing an Integrated Development Environments (IDEs) 
to lead engineers to automatically apply refactorings as such exist in others object-oriented languages.
 We describe a tool, implemented as an Eclipse \textit{plug-in} designed to fulfill exactly this need. 
This tool supports 16 refactorings that are heavily inspired by refactorings well known in the literature and yet keeping them in synch with the underlying source code. 
Our tool involves three steps: (\textit{i}) reverse engineering, (\textit{ii}) refactoring by means of KDM,
 and (\textit{iii}) forward engineering. 
The first step relies on transforming the source code into a KDM's instance. 
The second step relies on applying refactorings into the KDM's instance. 
The third step consists of generating the refactored source-code.


%ABSTRACT 2
%Refactoring is recognized as an essential practice in the context of evolutionary and agile software development.
%Nowadays, with the advent of Architecture-Driven Modernization (ADM) it has become a good alternative for refactoring systems in a Model-Driven Architecture (MDA) way by using its metamodel - Knowledge Discovery Metamodel (KDM). Nevertheless, although ADM provides the process for refactoring legacy systems by means of KDM, is so far missing an Integrated Development Environments (IDEs) to lead engineers to automatically apply refactorings as such exist in others object-oriented languages. We describe a tool, implemented as an Eclipse \textit{plug-in} designed to fulfill exactly this need. This tool supports 16 refactorings that are heavily inspired by refactorings well known in the literature. Our tool involves three steps: (\textit{i}) reverse engineering, (\textit{ii}) fully automated application of the refactoring by means of KDM, and (\textit{iii}) forward engineering. The first step relies on transforming the source code into a KDM's instance. The second step relies on applying refactorings into the KDM's instance. The third step consists of generating the refactored source-code.

\end{abstract}
\section{Introduction\label{sec:introduction}}
 Architecture-Driven Modernization (ADM) which has been proposed by OMG (Object Management Group) and advocates conducting refactoring following the principles of MDA (Model-Driven Architecture)~\cite{Ulrich:2010:IST:1841736} has become a great candidate since it aims to promote \textit{(i)} portability, \textit{(ii)} interoperability and \textit{(iii)} reusability. According to the OMG the most important artifact provided by ADM is the KDM metamodel, which is a vertical and horizontal standard metamodel that represents all elements of the existing information  technology architectures. KDM is structured in a hierarchy of four layers; \textit{Infrastructure Layer}, \textit{Program Elements Layer}, \textit{ Runtime Resource Layer} and \textit{Abstractions Layer}. We are specially interested in the \textit{Program Elements Layer} because it defines the ``Code'' packages which is widely used by our tool. It defines a set of metaclasses that represents the common elements in the source code supported by different programming languages such as: (\textit{i}) ``ClassUnit'' and ``InterfaceUnit'' which represent classes and interface, respectively, (\textit{ii}) ``StorableUnit'' which illustrates attributes and (\textit{iii}) ``MethodUnit'' to represent methods, etc.

On the other hand, refactoring has been known and highly used both industrially and academically. It is a form of transformation that was initially defined by Opdyke [ref] in the discipline of Object Oriented (OO) Programming as ``a change made to the internal structure of the software while preserving its external behavior at the same level of abstraction''. Nowadays it is possible to identify several catalogs of refactoring for different languages. The most complete and influential was published by Fowler in [6] for refactoring of Java code. 
However, while software reengineers would greatly benefit from the possibility to assess different choices, in practice they mostly rely on experience or intuition because of the lack of approaches providing comparison between possible variations of a change. 
Therefore, software reengineers do not have the possibility to easily apply analyses on different source-code version branches of a system and compare them to pick up the most adequate changes. In this context, the motivations for moving from software refactoring  to model refactoring are: 
(\textit{i}) a model provides an abstract view of the system, hence, visualizations of the structural changes required are easier, 
(\textit{ii}) problems uncovered at the design-level can be improved directly on the model and 
(\textit{iii}) using refactoring in high abstract level can allow the software engineer to explore alternate design paths in much cheaper than software refactoring.

To overcome the described problems, we devised a \textit{plug-in} on the top of the Eclipse Platform named KDM-RE (\textbf{K}nowledge \textbf{D}iscovery \textbf{M}odel-\textbf{R}efactoring \textbf{E}nvironment). Firstly, this \textit{plug-in} provides a model-driven refactoring based on KDM models. Secondly, it also supplies a multiple versions of a system at level models (KDM), enabling to the engineer  to work interactively on multiple models and to explore alternate refactoring path. Finally, after the best alternate refactoring path is chosen by the engineer, a forward engineering is performed and the source code of the refactored target system is generated again. Notice that KDM-RE supports 16 refactorings that are heavily inspired by the refatorings given by Fowler~\cite{aqui}. These refactorings are split up into three groups. The first group named \textbf{Moving Features Between Objects} which consists of relatively simple refactorings such as moving and creating features. The second group called \textbf{Organizing Data} is a set of refactorings to be applied in order to make working with data easier. The third group is named \textbf{Dealing With Generalization} and it represents refactorings to mostly dealing with moving methods and fields around a hierarchy of inheritance. This paper is organized as followed: Section 2 provides information related to KDM-RE - Section 3 the architecture of KDM-RE is depicted - in Section 4 there are related works and in Section 5 we conclude the paper with some remarks and future directions.







%\section{Motivation\label{sec:motivation}} 
% The motivation is the following: (\textit{i}) there is neither a repository, in which the CFs/CFFs can be stored by domain engineer in order to share it, nor an infrastructure where the CFs can be reused as easy as possible in order to disseminate them; (\textit{ii}) most of the CFs found apply white-box reuse strategies - thus it is important provides a way to assist the instantiation of these CFs using models.     

%(\textit{i}) provide an infrastructure to support the reuse of CFs, (\textit{ii}) allowing application engineers to download CF members to be reused in their applications and (\textit{iii}) allowing application engineers to conduct the reuse of CFs in a model-based fashion.  


%In the literature, it is possible find out a large number of researches related to CFF (COLOCAR REF). In these researches a large number of CFF have been devised (e.g., Security, Persistence, Distribution, etc). Thus, it is important develop an environment in which allows engineers share, manage, and provide full cycle of reuse of these CFF. It is also important supplies a graphical way to allow the engineer examines and reuses the features available in one CFF. For instance, the engineer could reuse piece of code of the CFF just selecting a set of features through a feature model.

%Therefore, this environment would increases the quality of the base applications that are developed in terms of modularity, reuse, maintainability. The efficiency will be achieved through the use of techniques related to reuse of analysis, design and code. The quality of the final applications must also be improved with the reuse, since the artifacts have already been tested and approved.  


%For instance, one engineer has developed a CFF and would like to share it in oder to let another engineer reuse his CFF.  CFF has beed developed and  engineer developed a CFF Moreover, it is also important supplies a graphical way to allow the engineer examines and reuses the features available in one CFF. For instance, the engineer could reuse piece of code of the CFF by selecting a set of features in a feature model. % if them fulfill the base application.  the features available fulfilled the base application the engineer would reuse such features.  %whether the features available in one CFF fulfill the base application requirements. 
%To the best of our knowledge there are neither an approach nor a tool with these functionality. In order to overcome such limitations we have put forward a plug-in, in which is described in the next sections.	 

%However, to the best of our knowledge, there is neither an approach nor a tool that allows share, management and also provides full cycle of reuse of the CFF. Moreover, it is important a tool that supplies a graphical way to allow the engineer examines previously if the features available in one CFF fulfill the application requirements. %Moreover, it is important that there is way to allow the engineer verify whether the variability/features available in the CFF meet the application requirements that must be developed.  

%Afterwards, the features have been examined and chosen by the engineer it is likewise important provides a way to download only the artifacts that have been chosen.    

%Thus, the CFF can be used during a development process easier and can be provided major computer support to facilitate such as task.     

%It is important that there is a way to examine whether the variability / features available in the family of FTs meet the application requirements that must be developed. In order for families of FTs to be properly used during a development process, and a major computer support to facilitate this task.


% which implements complete cycle of reuse of the CFF. 
%\textbf{Using this plug-in the CFF can be reused in a controlled way. Thus, during the reuse to increase the quality of the developed applications. Furthermore, any manipulation in the CFF can be made through the intermediary of models, so we intend to raise the level of abstraction in which the engineer works.}


%Thus, the objective of this paper is describes this plug-in. 

%Embora v�rios autores j� tenham trabalhado com FTs (Couto et al., 2005; Hanenberg et al., 2004; Huang et al., 2004; Shah e Hill, 2004; Constantinides e Elrad, 2001; Pinto et al., 2002; Rashid e Chitchyan, 2003; Soares et al., 2006; Vanhaute et al., 2001; Huang et al., 2004; Mortensen e Ghosh, 2006a; Mortensen e Ghosh, 2006b; Soares, 2004), n�o � encontrado na literatura o projeto de um reposit�rio de FTs e t�cnicas para busca de um determinado FT que atenda aos requisitos de uma aplica��o. � importante que exista uma forma de averiguar se as variabilidades/caracter�sticas dispon�veis na fam�lia de FTs atendem aos requisitos da aplica��o que deve ser desenvolvida. Para que fam�lias de FTs sejam adequadamente usadas durante um processo de desenvolvimento, � importante um apoio computacional que facilite essa tarefa.

\section{Background\label{sec:KDM-RE}} 

This section introduces the basic concepts of Architecture-Driven Modernization, Knowledge-Discovery Meta-model, and refactorings. 

\subsection{ADM and KDM}

OMG defined ADM initiative~\cite{1686216} which advocates carrying out the reengineering process considering MDA principles. In other words, ADM is the concept of modernizing existing systems with a focus on all aspects of the current systems architecture and the ability to transform current architectures to target architectures by using all principles of MDA~\cite{Ulrich:2010:IST:1841736}.

To perform a system modernization, ADM introduces Knowledge Discovery meta-model (KDM). KDM is an OMG specification adopted as ISO/IEC 19506 by the International Standards Organization for representing information related to existing software systems. According to P\'{e}rez-Castillo et al.,~\cite{1686216} the goal of the KDM standard is to define a meta-model to represent all the different legacy software artifacts involved in a legacy information system (e.g. source code, user interfaces, databases, etc.). The KDM provides a comprehensive high-level view of the behavior, structure and data of legacy information systems by means of a set of meta-models. The main purpose of the KDM specification is not the representation of models related strictly to the source code nature such as Unified Modeling Language (UML). While UML can be used to mainly to visualize the system ``as-is'', an ADM-based process using KDM starts from the different legacy software artifacts and builds higher-abstraction level models in a bottom-up manner through reverse engineering techniques.

As outlined before, the KDM consists of four abstraction layers: (\textit{i}) \emph{Infrastructure Layer}, (\textit{ii}) \emph{Program Elements Layer}, (\textit{iii}) \emph{Runtime Resource Layer}, and (\textit{iv}) \emph{Abstractions Layer}. Each layer is further organized into packages, as can be seen in Figure~\ref{fig:layersKDM}. Each package defines a set of meta-model elements whose purpose is to represent a certain independent facet of knowledge related to existing software systems.  We are specially interested in the \textit{Program Elements Layer} because it defines the Code and Action packages which are widely used by our catalogue. The Code package defines a set meta-classes that represents the common elements in the source code supported by different programming languages. In Table~\ref{tab:mappingCodeToKDM} is depicted some of them. This table identifies KDM meta-classes possessing similar characteristics to the static structure of the source code. Some meta-classes can be direct mapped, such as Class from object-oriented language, which can be easily mapped to the \texttt{ClassUnit} meta-class from KDM.

\begin{figure}[t]
\centering
  % Requires \usepackage{graphicx}
  \includegraphics[scale=0.55]{Figure/Layers_packages_and_separations_of_concerns_in_KDM}
\caption{Layers, packages, and separation of concerns in KDM (Adapted from~\cite{OMGADM})}
\label{fig:layersKDM}
\end{figure}


%TabelaPAraUsarNoVEM
\begin{table}[!h]
\caption{Meta-classes for Modeling the Static Structure of the Source-code}
\label{tab:mappingCodeToKDM}
\centering
  % Requires \usepackage{graphicx}
  \includegraphics[scale=1]{Figure/TabelaPAraUsarNoVEM}
\end{table}

\subsection{Refactoring and Model-Driven Refactoring}

In the area of object-oriented programming, refactorings are the technique of choice for improving the structure of existing code without changing its external behavior~\cite{refactImpro}. They have proved to be useful to improve the quality attributes of source code, and thus, to increase its maintainability. Nowadays, there are researches been carried out about apply refactoring in model instead of source code\cite{Ulrich:2010:IST:1841736}. Unfortunately, no catalogue of refactorings for the KDM specification exists. Nevertheless, although ADM provides the process for refactoring legacy systems by means of KDM, there is a lack of an Integrated Development Environment (IDE) to lead engineers to apply refactorings as such exist in others object-oriented languages. In the same direction, Model-Driven Refactoring is a special kind of model transformation that allows us to improve the structure of the model while preserving its internal quality characteristics. Model-driven refactoring is a considerably new area of research which still needs to reach the level of maturity attained by source code refactoring~\cite{ModelDrivenRefactoring}. 

Available object-oriented refactoring catalogues are not reusable as they are, because the KDM follow the MDA. This forces developers to create they own refactorings to be applied into models, i.e., they neither follow any catalogue nor use any kind of dedicated support. As a result, and due to the potential  complexity of model-driven refactoring, manual modifications into the models may lead to unwanted
side-effects and result in a tedious and error-prone maintenance process. In this sense, the main contribution of this paper is the provision of an IDE to lead engineers to apply refactorings in KDM, which are based on well known refactorings\cite{refactImpro}. The IDE as well as the adapted catalogue are based on our experience as model-driven engineering. %Also, the catalogue herein provided is heavily inspired by the refactorings given by Fowler~\cite{refactImpro}.



%________________________________________________________________________

\section{Refactoring for KDM by means of KDM-RE}\label{sec:refactoring_kdm_kdm_re}

This sections describes a \textit{plug-in} on the top of the Eclipse Platform named \textbf{K}nowledge \textbf{D}iscovery \textbf{M}odel-\textbf{R}efactoring \textbf{E}nvironment (KDM-RE). 
In Figure~\ref{fig:interface} we depicted the main window of our \textit{plug-in}. 
For explanation purpose, we have identified main regions, i.e., \textcircled{a}, and \textcircled{b}.
It supports 17 refactorings adapted to KDM. These refactorings are based on some fine-grained refactorings proposed by Fowler~\cite{refactImpro}. All the adapted refactorings are shown in Table~\ref{tab:adaptedRefactoring}. We chose the Fowler's refactorings once them are well known, basic and fine-grained refactorings. Please, not that KDM-RE uses MoDisco\footnote{\url{http://www.eclipse.org/MoDisco/}} once it provides an extensible framework to transform an specific source-code to KDM models.


%TabelaPAraUsarNoVEM
\begin{table}[!h]
\caption{Refactorings Adapted to KDM}
\label{tab:adaptedRefactoring}
\centering
  % Requires \usepackage{graphicx}
  \includegraphics[scale=0.67]{Figure/catalogue2}
\end{table}

\begin{figure}[!ht]
\centering
  % Requires \usepackage{graphicx}
  \includegraphics[width=14cm, height=6.8cm]{figure/ScreenShot_with_UML2}
\caption{KDM-RE's Interface}
\label{fig:interface}
\end{figure}
 
%Thus, suppose that a KDM model has already been instantiated. 
%All the steps to how obtain a KDM instance are explained further. 
%In order to assist the refactorings we extended the KDM's model browser provided by MoDisco. 

In Figure~\ref{fig:interface} is presented just a snippet of KDM-RE. Starting from the popup menu named ``Refactoring KDM'', in this model browser, see Figure~\ref{fig:interface}\textcircled{a}, either the software developer or software modernizer can interact with the KDM model and choose which refactoring must be carried out in the KDM.
%
%We added a popup menu named ``Refactoring KDM'' in this model browser, see Figure~\ref{fig:interface}\textcircled{a}.
%By using this menu the engineer can interact with the KDM model and choose which refactoring must be carried out in the KDM.
In the region \textcircled{a} can be seen all 17 refactorings that have been implemented in KDM-RE. 
For illustration purposes only we drew rectangles to separate the refactorings into three groups. 
The black rectangle represents refactorings that deal with generalization, the blue rectangle stand for refactorings to organize data and the red one symbolize refactoring to assist the moving features between objects.


The region \textcircled{b} on Figure~\ref{fig:interface} shows an UML class diagram that can be used before to apply some refactorings in order to assist the engineer to decide where to apply the refactorings. This UML class diagram also can be useful as the engineer performs the refactorings in KDM model, i.e., changes are reproduced on the fly in a class diagram.
We claim that the latter use of this UML class diagram is important once it provides an abstract view of the system, hence, the engineer/modernizer can visually check the system's changes after applying a set of refactorings. 
In addition, in the context of modernization of a legacy system usually the source-code is the only available artifact of that legacy system. 
Therefore, creating an UML class diagram makes, both the legacy system and the generated software to have a new type of artifact (i.e., UML class models), improving their documentation.

%KDM-RE also supplies a multiple versions of a system at level models (KDM) which allows the engineer to work interactively on multiple models and to explore alternate refactoring path. As can see in the region \textcircled{c} (see Figure~\ref{fig:interface}), the engineer must select a KDM file, then he must right-click on the mouse to appear a popup menu named ``Versions''. By releasing the mouse on this menu, three options is shown: (\textit{i}) List of changes, (\textit{ii}) Delete version and (\textit{iii}) Create a child version. The last option create a copy of the KDM file - then the engineer can explore another refactoring path. The second option delete a specific version - first option shows all changes that have been realized in a current KDM file, all changes are depicted in an Eclipse View, as shown in region \textcircled{d}. In this View it is possible to visualize the author that have committed the changes, the project and all refactorings realized.

\subsection{Case Study}

In this section, we motivate KDM-RE by analyzing an example. This example is a small part of the university domain.  Figure~\ref{fig:interface}~\textcircled{b} (left side) shows a class diagram used for modeling a small part of the
university domain. In an university there are several Persons, more specifically Professors, their Assistants, and Students. Each Person has RG, CPF, and address (of type String). Moreover, classes Professor, Assistant, and Student have an attribute name of type String each. Pretend that either the software modernizer or the software developer found out by looking at the UML class diagram (see Figure~\ref{fig:interface}\textcircled{b} left side) this redundantly, i.e., equal attributes in sibling classes. Therefore, he/she must apply the refactoring ``Pull Up Field'. Similarly, he/she also found out by looking at the UML class diagram that one class is doing work that should be done by two or more. For example, he/she found that the attributes RG and CPF should be modularized to a class. Similarly, it is necessary to provide more information about they address, such as number, city, country, etc. Therefore, he/she must apply the refactoring ``Extract Class'' to the attributes ``RG'', ``CPF'' and ``rua''. Due space limitation it is depicted just the extraction of the attributes ``RG'' and ``CPF''.  The first step is to select the meta-class that he/she identified as a bad smell, i.e., the meta-class to be extracted into a separate one.  This step is illustrated in Figure~\ref{fig:wizard}(a). 

\begin{figure}[!ht]
\centering
  % Requires \usepackage{graphicx}
  \includegraphics[scale=0.6]{figure/Wizard2}
\caption{Extract Class Wizard}
\label{fig:wizard}
\end{figure}

After selecting the meta-class, a right-click opens the context menu where the refactoring is accessible. After the click, the system displays the ``RefactoringWizard'' to the engineer, Figure~\ref{fig:wizard}(b) depicts the Extract Class Wizard. In this wizard, the name of the new meta-class can be set. Also a preview of all detected \texttt{StorableUnits} and \texttt{MethodUnits} that can be chosen to put into the new meta-class. Further, the engineer can select if either the new meta-class will be a top level meta-class or a nested meta-class. The engineer also can select if the KDM-RE must create instances of \texttt{MethodUnits} to represent accessors methods (gets and sets). Finally, the engineer can set the name of the \texttt{StorableUnit} that represent the link between the two meta-classes (the old meta-class and the new one). After all of the required inputs have been made, the engineer can click on the button ``Finish'' and the refactoring ``Extract Class'' is performed by KDM-RE. 

As can be seen in Figure~\ref{fig:wizard}(c) a new instance of \texttt{ClassUnit} named ``Document'' was created - two \texttt{StorableUnit} from ``Pessoa'', i.e., ``rg'' and ``CPF'' were moved to the new \texttt{ClassUnit} - instances of \texttt{MethodUnits} were also created to represent the gets and sets. In addition, the instance of \texttt{ClassUnit} named ``Pessoa'' owns a new instance of \texttt{StorableUnit} that represent the link between both \texttt{ClassUnits}. Due space limitation the other \texttt{StorableUnits} of \texttt{ClassUnit} named ``Pessoa'' are not shown in Figure~\ref{fig:wizard}(c). After the engineer realize the refactorings, an UML class diagram is created on the fly to mirror graphically all changes performed in the KDM model, see Figure~\ref{fig:interface}\textcircled{b} right side. %Moreover, the KDM-RE creates/updates a tracking log to show the historic of all changes performed in the system, as can be see in Figure~\ref{fig:interface}\textcircled{d}.                     

%\section{Architecture\label{sec:architecture}}
 %In Figure~\ref{fig:architecture} is depicted the architecture of KDM-RE which is split in three layers. As shown in this figure, the first layer of our \textit{plug-in} is the Core Framework. This layer represents that  we devised the  \textit{plug-in} on the top of the Eclipse Platform. In this layer it is also possible to see that we  used both Java and Groovy as programming language. Moreover, this layer contains Eclipse Plugins on which our tool is based on, such as MoDisco and EMF. We used MoDisco\footnote{http://www.eclipse.org/MoDisco/} that is an extensible framework to develop model-driven tools to support use-cases of existing software modernization and provides an Application Programming Interface - (API) to easily access the KDM model. Also, Eclipse Modeling Framewokr (EMF)\footnote{http://www.eclipse.org/modeling/emf/} was used to load and navigate KDM models that were generated with MoDisco. 

The second layer, the Tool Core, is where all refactorings provided by our \textit{plug-in} were implemented. KDM-RE works intensively with KDM models, which are XML files. Therefore, we use Groovy to handle those types of files because of the simplicity of its syntax and fully integrated with Java. KDM-RE also provides a way to create multiple versions of the KDM file to allow the engineer  to assess different refactorings in the same system. In order to optimizes memory usage of multiple versions for large models and enabling to work interactively on multiple models our \textit{plug-in} persists these models in a MongoDB. We chose MongoDB since it provides a high performance. After the engineer to choose a version refactored a forward engineering is carried out and the source code of the modernized target system is generated again. Finally, the top layer is the Graphical User Interface (GUI) that consists of a set of SWT windows with several options to perform the refactorings based on the KDM model.


\begin{figure}[!ht]
\centering
  % Requires \usepackage{graphicx}
  \includegraphics[scale=0.8]{figure/Arquitetura}
\caption{Architecture of KDM-RE}
\label{fig:architecture}
\end{figure}  

\section{Related Work\label{sec:related}}
 Borger et al.~\cite{Boger2002} developed a plug-in for the CASE tool ArgoUML that support UML model-based refactorings. The refactoring of class, states and activities is possible, allowing the user to apply refactorings that are not simple to apply at source code level.
Van Gorp et al.~\cite{Gorp03towardsautomating} proposed a UML profile to express pre and post conditions of source code refactorings using Object Constraint Language (OCL) constraints. The proposed profile allows that a CASE tool: (\textit{i}) verify pre and post conditions for the composition of sequences of refactorings; and (\textit{ii}) use the OCL consulting mechanism to detect bad smells such as crosscutting concerns.  Reimann et al.~\cite{Models2010} present an approach for EMF model refactoring. They propose the definition of EMF-based refactoring in a generic way. Another approach for EMF model refactoring is presented in~\cite{ModelsEMFREfactorin}. They propose EMF Refactor~\footnote{http://www.eclipse.org/emf-refactor/}, which is a new Eclipse incubation project in the Eclipse Modeling Project consisting of three main components. Besides a code generation module and a refactoring application module, it comes along with a suite of predefined EMF model refactorings for UML and Ecore models.

%The differential of our approach described herein in relation to the other is the proposal to move from software refactoring to model-driven refactoring by means of KDM, which is a platform and language independent meta-model. 

\section{Concluding Remarks\label{sec:conclusion}}
 In this paper is presented the KnowDIME to support the refactoring of legacy systems based on ADM, which uses the KDM standard. It follows the theory of the horseshoe modernization model, which is threefold: (\textit{i}) \textbf{Reverser Engineering}, (\textit{ii}) \textbf{Reestructuring} and  (\textit{iii}) \textbf{Forward Engineering}. 

Firstly, all the artefacts of the legacy system must be transformed into PSMs by statically analyzing the legacy source code. Still in the first step, these PSMs are integrated into a KDM model, i.e., a PIM model, through a M2M transformation implemented using ATL. Secondly, the KnowDIME applies a set of model refactorings and model optimizations in this PIM. Afterwards, KnowDIME executes a set of M2M transformation taking as input the PIM and producing as output a model conforming to the KDM models into UML meta model. Finally, an improved system is obtained from this UML by means of a set of M2C transformation; additionally, the generated code can be complemented by the software engineer in accordance with the more detailed specifications of the application business rules, such as the implementation of specific behaviors and features not covered by the code generation.

We believe that KnowDIME makes a contribution to the challenges of Software Engineering which focuses on mechanisms to support the automation of software refactoring process. Long term future work involves conducting experiments to evaluate the level of maintenance provided by KnowDIME. It is worth highlighting that KnowDIME is open source and it can be downloaded at\textit{~www.dc.ufscar.br/$\sim$valter/crossfire}.


\section{Acknowledgements}~\label{sec:ack}
 The authors would like to thank CNPq for Processes 241028/2012-4 and FAPESP for Process 2012/05168-4.

\bibliographystyle{apalike}
\bibliography{referencias}

\end{document}
